%%%%%%%%%%%%%%%%%%%%%%%%%%%%%%%%%%%%%%%%%
%                                       %
%       ET Programming Language         %
%               Max Base                %
%                                       %
%    https://github.com/ET-Lang/Book    %
%                                       %
%         Based on a template           %
%                                       %
%%%%%%%%%%%%%%%%%%%%%%%%%%%%%%%%%%%%%%%%%
\documentclass[11pt,fleqn]{book}
\usepackage[top=3cm,bottom=3cm,left=3.2cm,right=3.2cm,headsep=10pt,letterpaper]{geometry}
\usepackage{listings}
\usepackage{color}
\usepackage[osf,sc]{mathpazo}
\usepackage{amsmath}
\usepackage{listings}
\usepackage{xcolor}
\definecolor{ocre}{RGB}{52,177,201}
\usepackage{avant}
\usepackage{mathptmx}
\usepackage{microtype}
\usepackage[utf8]{inputenc}
\usepackage[T1]{fontenc}
\usepackage{amsthm}
\usepackage[style=alphabetic,sorting=nyt,sortcites=true,autopunct=true,babel=hyphen,hyperref=true,abbreviate=false,backref=true,backend=biber]{biblatex}
\addbibresource{bibliography.bib}
\defbibheading{bibempty}{}
%----------------------------------------------------------------------------------------
%	VARIOUS REQUIRED PACKAGES
%----------------------------------------------------------------------------------------

\usepackage{titlesec} % Allows customization of titles

\usepackage{graphicx} % Required for including pictures
\graphicspath{{Pictures/}} % Specifies the directory where pictures are stored
% \graphicspath{{Plots/}}
\usepackage{lipsum} % Inserts dummy text

\usepackage{tikz} % Required for drawing custom shapes

\usepackage[english]{babel} % English language/hyphenation

\usepackage{enumitem} % Customize lists
\setlist{nolistsep} % Reduce spacing between bullet points and numbered lists

\usepackage{booktabs} % Required for nicer horizontal rules in tables

\usepackage{eso-pic} % Required for specifying an image background in the title page

%----------------------------------------------------------------------------------------
%	MAIN TABLE OF CONTENTS
%----------------------------------------------------------------------------------------

\usepackage{titletoc} % Required for manipulating the table of contents

\contentsmargin{0cm} % Removes the default margin
% Chapter text styling
\titlecontents{chapter}[1.25cm] % Indentation
{\addvspace{15pt}\large\sffamily\bfseries} % Spacing and font options for chapters
{\color{ocre!60}\contentslabel[\Large\thecontentslabel]{1.25cm}\color{ocre}} % Chapter number
{}  
{\color{ocre!60}\normalsize\sffamily\bfseries\;\titlerule*[.5pc]{.}\;\thecontentspage} % Page number
% Section text styling
\titlecontents{section}[1.25cm] % Indentation
{\addvspace{5pt}\sffamily\bfseries} % Spacing and font options for sections
{\contentslabel[\thecontentslabel]{1.25cm}} % Section number
{}
{\sffamily\hfill\color{black}\thecontentspage} % Page number
[]
% Subsection text styling
\titlecontents{subsection}[1.25cm] % Indentation
{\addvspace{1pt}\sffamily\small} % Spacing and font options for subsections
{\contentslabel[\thecontentslabel]{1.25cm}} % Subsection number
{}
{\sffamily\;\titlerule*[.5pc]{.}\;\thecontentspage} % Page number
[] 

%----------------------------------------------------------------------------------------
%	MINI TABLE OF CONTENTS IN CHAPTER HEADS
%----------------------------------------------------------------------------------------

% Section text styling
\titlecontents{lsection}[0em] % Indendating
{\footnotesize\sffamily} % Font settings
{}
{}
{}

% Subsection text styling
\titlecontents{lsubsection}[.5em] % Indentation
{\normalfont\footnotesize\sffamily} % Font settings
{}
{}
{}
 
%----------------------------------------------------------------------------------------
%	PAGE HEADERS
%----------------------------------------------------------------------------------------

\usepackage{fancyhdr} % Required for header and footer configuration

\pagestyle{fancy}
\renewcommand{\chaptermark}[1]{\markboth{\sffamily\normalsize\bfseries\chaptername\ \thechapter.\ #1}{}} % Chapter text font settings
\renewcommand{\sectionmark}[1]{\markright{\sffamily\normalsize\thesection\hspace{5pt}#1}{}} % Section text font settings
\fancyhf{} \fancyhead[LE,RO]{\sffamily\normalsize\thepage} % Font setting for the page number in the header
\fancyhead[LO]{\rightmark} % Print the nearest section name on the left side of odd pages
\fancyhead[RE]{\leftmark} % Print the current chapter name on the right side of even pages
\renewcommand{\headrulewidth}{0.5pt} % Width of the rule under the header
\addtolength{\headheight}{2.5pt} % Increase the spacing around the header slightly
\renewcommand{\footrulewidth}{0pt} % Removes the rule in the footer
\fancypagestyle{plain}{\fancyhead{}\renewcommand{\headrulewidth}{0pt}} % Style for when a plain pagestyle is specified

% Removes the header from odd empty pages at the end of chapters
\makeatletter
\renewcommand{\cleardoublepage}{
\clearpage\ifodd\c@page\else
\hbox{}
\vspace*{\fill}
\thispagestyle{empty}
\newpage
\fi}

%----------------------------------------------------------------------------------------
%	THEOREM STYLES
%----------------------------------------------------------------------------------------

\usepackage{amsmath,amsfonts,amssymb,amsthm} % For math equations, theorems, symbols, etc

\newcommand{\intoo}[2]{\mathopen{]}#1\,;#2\mathclose{[}}
\newcommand{\ud}{\mathop{\mathrm{{}d}}\mathopen{}}
\newcommand{\intff}[2]{\mathopen{[}#1\,;#2\mathclose{]}}
\newtheorem{notation}{Notation}[chapter]

%%%%%%%%%%%%%%%%%%%%%%%%%%%%%%%%%%%%%%%%%%%%%%%%%%%%%%%%%%%%%%%%%%%%%%%%%%%
%%%%%%%%%%%%%%%%%%%% dedicated to boxed/framed environements %%%%%%%%%%%%%%
%%%%%%%%%%%%%%%%%%%%%%%%%%%%%%%%%%%%%%%%%%%%%%%%%%%%%%%%%%%%%%%%%%%%%%%%%%%
\newtheoremstyle{ocrenumbox}% % Theorem style name
{0pt}% Space above
{0pt}% Space below
{\normalfont}% % Body font
{}% Indent amount
{\small\bf\sffamily\color{ocre}}% % Theorem head font
{\;}% Punctuation after theorem head
{0.25em}% Space after theorem head
{\small\sffamily\color{ocre}\thmname{#1}\nobreakspace\thmnumber{\@ifnotempty{#1}{}\@upn{#2}}% Theorem text (e.g. Theorem 2.1)
\thmnote{\nobreakspace\the\thm@notefont\sffamily\bfseries\color{black}---\nobreakspace#3.}} % Optional theorem note
\renewcommand{\qedsymbol}{$\blacksquare$}% Optional qed square

\newtheoremstyle{blacknumex}% Theorem style name
{5pt}% Space above
{5pt}% Space below
{\normalfont}% Body font
{} % Indent amount
{\small\bf\sffamily}% Theorem head font
{\;}% Punctuation after theorem head
{0.25em}% Space after theorem head
{\small\sffamily{\tiny\ensuremath{\blacksquare}}\nobreakspace\thmname{#1}\nobreakspace\thmnumber{\@ifnotempty{#1}{}\@upn{#2}}% Theorem text (e.g. Theorem 2.1)
\thmnote{\nobreakspace\the\thm@notefont\sffamily\bfseries---\nobreakspace#3.}}% Optional theorem note

\newtheoremstyle{blacknumbox} % Theorem style name
{0pt}% Space above
{0pt}% Space below
{\normalfont}% Body font
{}% Indent amount
{\small\bf\sffamily}% Theorem head font
{\;}% Punctuation after theorem head
{0.25em}% Space after theorem head
{\small\sffamily\thmname{#1}\nobreakspace\thmnumber{\@ifnotempty{#1}{}\@upn{#2}}% Theorem text (e.g. Theorem 2.1)
\thmnote{\nobreakspace\the\thm@notefont\sffamily\bfseries---\nobreakspace#3.}}% Optional theorem note

%%%%%%%%%%%%%%%%%%%%%%%%%%%%%%%%%%%%%%%%%%%%%%%%%%%%%%%%%%%%%%%%%%%%%%%%%%%
%%%%%%%%%%%%% dedicated to non-boxed/non-framed environements %%%%%%%%%%%%%
%%%%%%%%%%%%%%%%%%%%%%%%%%%%%%%%%%%%%%%%%%%%%%%%%%%%%%%%%%%%%%%%%%%%%%%%%%%
\newtheoremstyle{ocrenum}% % Theorem style name
{5pt}% Space above
{5pt}% Space below
{\normalfont}% % Body font
{}% Indent amount
{\small\bf\sffamily\color{ocre}}% % Theorem head font
{\;}% Punctuation after theorem head
{0.25em}% Space after theorem head
{\small\sffamily\color{ocre}\thmname{#1}\nobreakspace\thmnumber{\@ifnotempty{#1}{}\@upn{#2}}% Theorem text (e.g. Theorem 2.1)
\thmnote{\nobreakspace\the\thm@notefont\sffamily\bfseries\color{black}---\nobreakspace#3.}} % Optional theorem note
\renewcommand{\qedsymbol}{$\blacksquare$}% Optional qed square
\makeatother

% Defines the theorem text style for each type of theorem to one of the three styles above
\newcounter{dummy} 
\numberwithin{dummy}{section}
\theoremstyle{ocrenumbox}


\newtheorem{theoremeT}[dummy]{Theorem}
\newtheorem{lemma}[dummy]{Lemma}
\newtheorem{observation}[dummy]{Observation}
\newtheorem{proposition}[dummy]{Proposition}
% \newtheorem{definition}[dummy]{Definition}
\newtheorem{claim}[dummy]{Claim}
\newtheorem{fact}[dummy]{Fact}
\newtheorem{assumption}[dummy]{Assumption}

\newtheorem{problem}{Problem}[chapter]
% \newtheorem{exercise}{Exercise}[chapter]
\theoremstyle{blacknumex}
\newtheorem{exampleT}{Example}[chapter]
\theoremstyle{blacknumbox}
\newtheorem{vocabulary}{Vocabulary}[chapter]
\newtheorem{definitionT}{Definition}[section]
\newtheorem{corollaryT}[dummy]{Corollary}
\theoremstyle{ocrenum}

%----------------------------------------------------------------------------------------
%	DEFINITION OF COLORED BOXES
%----------------------------------------------------------------------------------------

\RequirePackage[framemethod=default]{mdframed} % Required for creating the theorem, definition, exercise and corollary boxes

% Theorem box
\newmdenv[skipabove=7pt,
skipbelow=7pt,
backgroundcolor=black!5,
linecolor=ocre,
innerleftmargin=5pt,
innerrightmargin=5pt,
innertopmargin=5pt,
leftmargin=0cm,
rightmargin=0cm,
innerbottommargin=5pt]{tBox}

% Exercise box	  
\newmdenv[skipabove=7pt,
skipbelow=7pt,
rightline=false,
rightline=true,
topline=false,
bottomline=false,
backgroundcolor=ocre!10,
linecolor=ocre,
innerrightmargin=5pt,
innerrightmargin=5pt,
innertopmargin=5pt,
innerbottommargin=5pt,
leftmargin=0cm,
rightmargin=0cm,
linewidth=4pt]{eBox}	

% Definition box
\newmdenv[skipabove=7pt,
skipbelow=7pt,
rightline=false,
leftline=true,
topline=false,
bottomline=false,
linecolor=ocre,
innerleftmargin=5pt,
innerrightmargin=5pt,
innertopmargin=0pt,
leftmargin=0cm,
rightmargin=0cm,
linewidth=4pt,
innerbottommargin=0pt]{dBox}	

% Corollary box
\newmdenv[skipabove=7pt,
skipbelow=7pt,
rightline=false,
leftline=true,
topline=false,
bottomline=false,
linecolor=gray,
backgroundcolor=black!5,
innerleftmargin=5pt,
innerrightmargin=5pt,
innertopmargin=5pt,
leftmargin=0cm,
rightmargin=0cm,
linewidth=4pt,
innerbottommargin=5pt]{cBox}

% Creates an environment for each type of theorem and assigns it a theorem text style from the "Theorem Styles" section above and a colored box from above
\newenvironment{theorem}{\begin{tBox}\begin{theoremeT}}{\end{theoremeT}\end{tBox}}
\newenvironment{exercise}{\begin{eBox}\begin{exerciseT}}{\hfill{\color{ocre}\tiny\ensuremath{\blacksquare}}\end{exerciseT}\end{eBox}}				  
\newenvironment{definition}{\begin{dBox}\begin{definitionT}}{\end{definitionT}\end{dBox}}	
\newenvironment{example}{\begin{exampleT}}{\hfill{\tiny\ensuremath{\blacksquare}}\end{exampleT}}		
\newenvironment{corollary}{\begin{cBox}\begin{corollaryT}}{\end{corollaryT}\end{cBox}}	

%----------------------------------------------------------------------------------------
%	REMARK ENVIRONMENT
%----------------------------------------------------------------------------------------

\newenvironment{remark}{\par\vspace{10pt}\small % Vertical white space above the remark and smaller font size
\begin{list}{}{
\leftmargin=35pt % Indentation on the left
\rightmargin=25pt}\item\ignorespaces % Indentation on the right
\makebox[-2.5pt]{\begin{tikzpicture}[overlay]
\node[draw=ocre!60,line width=1pt,circle,fill=ocre!25,font=\sffamily\bfseries,inner sep=2pt,outer sep=0pt] at (-15pt,0pt){\textcolor{ocre}{R}};\end{tikzpicture}} % Orange R in a circle
\advance\baselineskip -1pt}{\end{list}\vskip5pt} % Tighter line spacing and white space after remark

%----------------------------------------------------------------------------------------
%	SECTION NUMBERING IN THE MARGIN
%----------------------------------------------------------------------------------------

\makeatletter
\renewcommand{\@seccntformat}[1]{\llap{\textcolor{ocre}{\csname the#1\endcsname}\hspace{1em}}}                    
\renewcommand{\section}{\@startsection{section}{1}{\z@}
{-4ex \@plus -1ex \@minus -.4ex}
{1ex \@plus.2ex }
{\normalfont\large\sffamily\bfseries}}
\renewcommand{\subsection}{\@startsection {subsection}{2}{\z@}
{-3ex \@plus -0.1ex \@minus -.4ex}
{0.5ex \@plus.2ex }
{\normalfont\sffamily\bfseries}}
\renewcommand{\subsubsection}{\@startsection {subsubsection}{3}{\z@}
{-2ex \@plus -0.1ex \@minus -.2ex}
{.2ex \@plus.2ex }
{\normalfont\small\sffamily\bfseries}}                        
\renewcommand\paragraph{\@startsection{paragraph}{4}{\z@}
{-2ex \@plus-.2ex \@minus .2ex}
{.1ex}
{\normalfont\small\sffamily\bfseries}}

%----------------------------------------------------------------------------------------
%	HYPERLINKS IN THE DOCUMENTS
%----------------------------------------------------------------------------------------

% For an unclear reason, the package should be loaded now and not later
\usepackage{hyperref}
\hypersetup{hidelinks,backref=true,pagebackref=true,hyperindex=true,colorlinks=false,breaklinks=true,urlcolor= ocre,bookmarks=true,bookmarksopen=false,pdftitle={Title},pdfauthor={Author}}

%----------------------------------------------------------------------------------------
%	CHAPTER HEADINGS
%----------------------------------------------------------------------------------------

% The set-up below should be (sadly) manually adapted to the overall margin page septup controlled by the geometry package loaded in the main.tex document. It is possible to implement below the dimensions used in the goemetry package (top,bottom,left,right)... TO BE DONE

\newcommand{\thechapterimage}{}
\newcommand{\chapterimage}[1]{\renewcommand{\thechapterimage}{#1}}

% Numbered chapters with mini tableofcontents
\def\thechapter{\arabic{chapter}}
\def\@makechapterhead#1{
\thispagestyle{empty}
{\centering \normalfont\sffamily
\ifnum \c@secnumdepth >\m@ne
\if@mainmatter
\startcontents
\begin{tikzpicture}[remember picture,overlay]
\node at (current page.north west)
{\begin{tikzpicture}[remember picture,overlay]
\node[anchor=north west,inner sep=0pt] at (0,0) {\includegraphics[width=\paperwidth]{\thechapterimage}};
%%%%%%%%%%%%%%%%%%%%%%%%%%%%%%%%%%%%%%%%%%%%%%%%%%%%%%%%%%%%%%%%%%%%%%%%%%%%%%%%%%%%%
% Commenting the 3 lines below removes the small contents box in the chapter heading
%\fill[color=ocre!10!white,opacity=.6] (1cm,0) rectangle (8cm,-7cm);
%\node[anchor=north west] at (1.1cm,.35cm) {\parbox[t][8cm][t]{6.5cm}{\huge\bfseries\flushleft \printcontents{l}{1}{\setcounter{tocdepth}{2}}}};
\draw[anchor=west] (5cm,-9cm) node [rounded corners=20pt,fill=ocre!10!white,text opacity=1,draw=ocre,draw opacity=1,line width=1.5pt,fill opacity=.6,inner sep=12pt]{\huge\sffamily\bfseries\textcolor{black}{\thechapter. #1\strut\makebox[22cm]{}}};
%%%%%%%%%%%%%%%%%%%%%%%%%%%%%%%%%%%%%%%%%%%%%%%%%%%%%%%%%%%%%%%%%%%%%%%%%%%%%%%%%%%%%
\end{tikzpicture}};
\end{tikzpicture}}
\par\vspace*{230\p@}
\fi
\fi}

% Unnumbered chapters without mini tableofcontents (could be added though) 
\def\@makeschapterhead#1{
\thispagestyle{empty}
{\centering \normalfont\sffamily
\ifnum \c@secnumdepth >\m@ne
\if@mainmatter
\begin{tikzpicture}[remember picture,overlay]
\node at (current page.north west)
{\begin{tikzpicture}[remember picture,overlay]
\node[anchor=north west,inner sep=0pt] at (0,0) {\includegraphics[width=\paperwidth]{\thechapterimage}};
\draw[anchor=west] (5cm,-9cm) node [rounded corners=20pt,fill=ocre!10!white,fill opacity=.6,inner sep=12pt,text opacity=1,draw=ocre,draw opacity=1,line width=1.5pt]{\huge\sffamily\bfseries\textcolor{black}{#1\strut\makebox[22cm]{}}};
\end{tikzpicture}};
\end{tikzpicture}}
\par\vspace*{230\p@}
\fi
\fi
}
\makeatother

\def\R{\mathbb{R}}
\newcommand{\cvx}{convex}
\begin{document}

%-----------------------------
% TITLE
%-----------------------------
\begingroup
\thispagestyle{empty}
\AddToShipoutPicture*{{\includegraphics[scale=1.32]{esahubble}}} \centering
\vspace*{5cm}
\par\normalfont\fontsize{35}{35}\sffamily\selectfont
\textbf{ET Programming Language}\\
%{\LARGE Introduction a new and modern system programming language}\par % Book title
\vspace*{1cm}
{\Huge Max Base}\par
\endgroup
%-----------------------------
% COPYRIGHT
%-----------------------------
\newpage
~\vfill
\thispagestyle{empty}

\noindent Copyright \copyright\ 2019 Max Base\\ 

\noindent \textsc{Max Base, Asrez Team}\\
\noindent \textsc{Asrez.com}\\

\noindent This research was done under the supervision of Dr. Pauline Barmby with the financial support of the MITACS Globalink Research Internship Award within a total of 12 weeks, from June 16th to September 5th of 2014.\\

\noindent \textit{First release, Jun 2019}
%-----------------------------
% TABLE
%-----------------------------
\chapterimage{head1.png}
\pagestyle{empty}
\tableofcontents
\pagestyle{fancy}
%-----------------------------
% CHAPTER 1
%-----------------------------
\chapterimage{head2.png}
\chapter{ET Programming Language}
\section{Introduction}

The ET language has been designed by Max Base in 2012, Which was later
developed by the Asrez team. They tried to design a modern, more appropriate language. As
simple as possible for humans.
Some features of the ET language:



    \begin{itemize}
        \item ET Language is a middle-level language. Programming languages ​​can be divided into three categories: high-level languages, middle level, low-level languages. (Table 1-1) The reason for the middle-level language is that it is also low because it is capable of being closely related to the hardware, like the assembly. And on the other side it is close to human expression and has simple commands and is readiable for humans then this's a feature of the high level languages.
        
        \item ET language is a flexible and powerful language that does not create any restrictions for the programmer, and you can take and create whatever you think.
        
        \item ET language is a portable language. This means that you can run the code written in a system on another system. Without causing trouble or trouble.
        
        \item ET language is a multi-platform language. This means that when you design an application for the Windows operating system. You can run it on another operating system, including Linux series.
        
        \item The ET programming language has the ability to embed on a platform for another platform.
        So you can design a program for Windows. You will not need to own a Windows operating system.
        And this is a unique and wonderful feature.
        
        \item The ET programming language has tiny keywords. This means that the number of keywords in this language is small. But this does not make you curious or can not produce any program.
        However, the number of keywords is not a reason for the strength and speed of the compiler. But these are the reasons for learning this programming language easier and faster. \begin{table}[]
            \centering
    \begin{tabular}{|lllll|}
\hline
    auto    & const & break & continue & return \\
    if  & for  & foreach & loop & while \\
    do  & default & switch & else & elseif \\
    elif    & extern & struct & enum & static \\
    goto & union & case & register & typedef\\
    assembly & import & ifdef & ifndef & define \\ \hline
    
    \end{tabular}
\end{table}


\pagebreak

    
    
    \item ET Language is a system language. which you can even design system programs. System programs are programs that allow you to exploit hardware and software communitication. Some system programs are: operating system / interpreter / compiler / database / word processor and assembler and ...
    
    \item There is a close connection between the ET and the assembly.
    This means that you can also insert assembly code into this language.
    Of course, consider that they are controlled by a separate processing engine.
    
    \item ET Language is a sensitive letter.
    This means that words in lowercase letters are different.
    So be sure to type the letters you type in the letters you type.
    
    \item The ET language instructions are included in the following features:
    
    \begin{itemize}
    \item There is no limit to the number of words per line.
    \item It is recommended to write just one command per line
    \item Ability to use; at the end of each instruction is optional
    \item Each command can be written in several lines and lines. (You can create spaces or empty space between your guides.

\item You can write comment in your code that you should use the principles of communication.   
    \end{itemize}

    

\end{itemize}




\section{Data Types}


The purpose of programming is to receive inputs and process them to produce an output.
Keep in mind that input values is the most important part of your work.


In the programming language, there are several types of data that we can use in the place where it is needed:
float, int, string, char, void, bool, null, ...

The type is to store individual char (such as 'a', 'b'), The type is for storing int for keeping Integer (such as 4, 32, 169) and the float type is used for decimal number such as 15.4, 64.26) And we will explain the type of void later.

But keep in mind that each of them has limitations.
So we have several types with the name int, each of which has a limit.
We can choose them according to the required range.

\pagebreak


\begin{table}[]
    \centering
    \begin{tabular}{|l|l|l|}
        \hline
        int8 & 8bit, 1byte & -127 until 128   \\ \hline
        uint8 & 8bit, 1byte & -127 until 128   \\ \hline

        int16 & 16bit, 2byte & -127 until 128   \\ \hline
        uint16 & 16bit, 2byte & -127 until 128   \\ \hline

        int32 & 32bit, 3byte & -127 until 128   \\ \hline
        uint32 & 32bit, 3byte & -127 until 128   \\ \hline

        int64 & 64bit, 4byte & -127 until 128   \\ \hline
        uint64 & 64bit, 4byte & -127 until 128   \\ \hline

        size & 64bit, 4byte & -127 until 128   \\ \hline

        float16 & 16bit, 2byte & -127 until 128   \\ \hline
        ufloat16 & 16bit, 2byte & -127 until 128   \\ \hline

        float32 & 32bit, 3byte & -127 until 128   \\ \hline
        ufloat32 & 32bit, 3byte & -127 until 128   \\ \hline

        char & 8bit, 1byte & -127 until 128   \\ \hline
        uchar & 16bit, 2byte & -127 until 128   \\ \hline

        mchar & 16bit, 2byte & -127 until 128   \\ \hline

        bool & 1bit & true or false \\ \hline
        null & 1bit & empty \\ \hline

    \end{tabular}
\end{table}
\label{List of data type(s)}







\section{Variable}


A variable is a name for memory words that we put data into. And we may change them and use them during the implementation of the program.
To refer to their value, we use the same name, which is the reason we name them so that they can be easily accessed.

To name variables, we can use characters from a to z or A to Z. As well as numbers and \_ char.

We do not have a number limit for the length of name of variables.

Note that the name of the variables can not start with the number. (The first letter of the name can not be a number)


\subsection{Define Variable}


As mentioned, variables are memory spaces. So, as the data has the type. We must also specify the type of variables.
So the variable definition method will be:

In this case, the variable type can be one of the values in Table 2

`VariableName dataType '...


\subsection{Story}

\subsection{Name}

The reason for naming ET for this time was initially related to the
concept of electronics and technology. But later on the suggestion of Javad Sabet the concept of
the name has been changed. This language also means the language of the earth (earth tongue).


\begin{definition}[Cone]
    A set $K \in \R^n$, when $x \in K $ implies $\alpha x \in K$.
\end{definition}

\begin{example}[ludcator function]


$\delta_c(x) = \begin{cases}
0 \quad  x \in C \\
+ \infty \quad elsewhere
\end{cases}$.\\
$dom \space \delta_c(x) = C$
\end{example}


\begin{definition}[III]
A function $f : \R^n \to \bar{\R}$ is %\cvx  if $\epi \space f $ is \cvx
\end{definition}


\begin{theorem}
$f : \R^n \to \bar{\R}$ is \cvx  $\iff$ $\forall x,y \in \R^n, \alpha \in (0,1), f(ax + (1-a)x) \le af(x) + (1-a)f(x)$.
\end{theorem}

\begin{lstlisting}[language=html]{test.py}
<html>
<head>
<title>Hello</title>
</head>
<body>
Hello
</body>
</html>
\end{lstlisting}


\end{document}
